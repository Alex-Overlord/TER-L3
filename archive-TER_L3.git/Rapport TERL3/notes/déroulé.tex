
\documentclass[11pt,letterpaper]{article}
\usepackage{fullpage}
\usepackage[top=1.5cm, bottom=3.5cm, left=1.5cm, right=1.5cm]{geometry}
\usepackage{amsmath,amsthm,amsfonts,amssymb,amscd}
\usepackage{french}
\usepackage[utf8]{inputenc}
\usepackage[T1]{fontenc}
\usepackage{lastpage}
\usepackage{enumerate}
\usepackage{fancyhdr}
\usepackage{mathrsfs}
\usepackage{xcolor}
\usepackage{graphicx}
\usepackage{sidecap}
\sidecaptionvpos{figure}{c}
\usepackage{listings}
\usepackage{hyperref}
\usepackage{cleveref}

% Edit these as appropriate
\newcommand\course{HLIN512I/HLIN601}
\newcommand\NetIDa{Groupe 1}           
\newcommand\NetIDb{L3 CMI Informatique}         

\pagestyle{fancyplain}
\headheight 35pt
\lhead{\NetIDa}
\lhead{\NetIDa\\\NetIDb}                 
\chead{\textbf{Étude des préférences dans l’expression\\de la quantification}}
\rhead{\course \\ Notes}
\headsep 1.5em

\begin{document}

\section{Suivi des réunions}

% Début 14/10/19
\subsection{Lundi 14/10/19}
Mettre les phrases manuellement (faire la génération automatique plus
tard, afin de garder assez de temps pour la diffusion et l'analyse statistique). \vspace{5mm}
    
\textbf{Questionnaire} :
\begin{itemize}
    \item La langue française est-il votre langue maternelle ?
    \item Quel est votre âge ?
    \item Quel est votre domaine d'étude (et/ou secteur d'activité) ?
\end{itemize} \vspace{5mm}

\textbf{Fin décembre} :
\begin{itemize}
    \item Avoir des pages de questions faites à la main
    \item Avoir une base de donnée
    \item Avoir un timer (chronométrer le temps de réponse)
\end{itemize} \vspace{5mm}
    
\textbf{Après Noël} : Faire en beaucoup mieux
\begin{itemize}
    \item Avoir un vrai accès au site (si on l'avait pas avant)
    \item Génération de phrases/questions
\end{itemize} \vspace{5mm}
    
\textbf{Site et Questionnaire} :
\begin{itemize}
    \item Faire la demande pour avoir un serveur accessible depuis l'extérieur
        
    \item Des phrases avec 2 quantificateurs :
    \begin{itemize}
        \item ?? (Pour tout)
        \item Exemples : Tous, Chaque, Les, Tous les, ...
        \item Existentiel (Il existe)
        \item Exemples : Un, Certains, Quelques, ...
        \item Et permuter les quantificateurs de places
    \end{itemize}
    \item Il faut des phrases ambiguës
\end{itemize} \vspace{5mm}

\textbf{Exemples de phrases} :
\begin{itemize}
    \item Les enfants prendront une pizza
    \item Est-ce qu'elle a des enfants ? Oui, un. (Un peut vouloir dire 'Pour tout', 'Il existe', '1')
\end{itemize} \vspace{5mm}

Voir l'article d'Arthur Capetier sur ses expériences audio. \vspace{5mm}
% Fin 14/10/19

% Début Jour 04/11/19
\newpage
\subsection{Jour 04/11/19}

\textbf{Changement sur les phrase}

\begin{itemize}
    \item Mettre la 11ème phrase avec \textit{Tout les touriste}
    \item Mettre la 10ème phrase \textit{Un travail}, la phrase n'est pas très ambigue.
\end{itemize} 

\vspace{5mm}
\textbf{Ecrire des phrases}
\begin{itemize}
    \item Faire varier \textit{Tout les} avec \textit{chaque} ect...
    \item exemple: Les mêmes phrase: \textit{Un etudiant doit avoir sa carte etudiant semble} être equivalent à \textit{Tout les etudiants}
\end{itemize}
Utiliser des quantificateur comme \textit{UN}, \textit{AU MOINS}... Et des universelle \textit{TOUS LES}, \textit{UN}, \textit{LE}.
%Fin Jour 04/11/19

% Début Jour 18/11/19
\newpage
\subsection{Jour 18/11/19}

\textbf{Avancement du site Web }
\begin{itemize}
\item Mettre le premier question (sur les vidéo surveillence) en dernier. En effet cela semble être une question qui n'a pas beaucoup d'ambiguïté en fonction de l'âge.
\item Ajouter les section  Droit, Santé, Science Humaine, Science, Lettre, Science Eco.
\item Stocker l'ordre de réponse et des question des utilisateur. Et les utiliser pour l'étude statistique.
\item Mettre le timer pour chronomètre les question.
\item Génération attention des mots
\item  Attention sur la diffusion des résultat, \textit{ne pas parler au résultat des autre mais les résultat par rapport au autre}.
\end{itemize}
\textbf{Rédaction du compte rendu}
\begin{itemize}
\item Commencer la bibliographie
\item Répartition des tache chacun doit prendre un partie et rédiger un plan détailler.
\item Dérouler du projet -> organisationnelle et mettre les statistiques a la fin du projet sur les donnée saisie du premier site
\item Ajout éventuellement d'un diagramme de Gantt.
\end{itemize}
% Fin Jour 18/11/19

% Modèle à copier
% Début Jour 04/12/19
\newpage
\subsection{Jour 04/12/19}
Regarder si le choix des mots pour symboliser un quantificateur, influence le sens que comprend le locuteur.
\vspace{5mm}
% Fin Jour 04/12/19

% Début Jour 18/12/19
\newpage
\subsection{Jour 18/12/19}
Pour le rapport:
\begin{itemize}
    \item Mettre un diagramme de Gantt
    \item Parler des articles que nous avons lu et analysé
\end{itemize}{} \vspace{5mm}
% Fin Jour 18/12/19

% Début --- Modèle à copier ---
% Début Jour JJ/MM/AA
\newpage
\subsection{Jour JJ/MM/AA}
Notes de la séance... \vspace{5mm}
\textbf{Liste} :
\begin{itemize}
    \item item
\end{itemize} \vspace{5mm}
% Fin Jour JJ/MM/AA
% Fin --- Modèle à copier ---
%-----------------------------------------------------------
\newpage

\section{BIBLIO}
\par Les mathématiques ne sont pas un langage, mais une connaissance. Il est clair
cependant que le langage naturel et le symbolisme jouent un rôle essentiel dans
l’activité mathématique et dans son apprentissage.

\vspace{5mm}
\par En mathématique, la quantification peut être définie comme l’ensemble des expressions utilisées pour formuler des propositions mathématiques dans le calcul des prédicats.

\vspace{5mm}
Le calcul des prédicats est une formalisation du langage des mathématiques.
\par Les symboles représentant les expressions en langage formel sont appelés des quantificateurs. Il en existe  deux types  de  quantification:  la quantification universelle et  la  quantification existentielle.

\vspace{5mm}
QUANTIFICATEURS UNIVERSELS :
\par La  quantification  universelle  peut  signifier  intuitivement  «attribuer  la  propriété  Q  à  toutes  les entités d’un ensemble P».(Mari et Retoré,  2016)
\par ==>  Le quantificateur TOUT :\par
- un  quantificateur  universel  du  premier  ordre,  il  assigne  une  propriété  à  chacun  des individus atomique (Dowty, 1987)\par
- incompatible avec une situation dans laquelle des individus particuliers satisfont la propriété (Brisson, 2013).\par
-Selon Brissonet Lasersohn, il n’est pas analysé comme un véritable quantificateur, mais plutôt comme un «réducteur de marge» (Lasersohn, 1999).Son emploi repose sur l’observation de l’existence d’une règle :\par         P(x)->Q(x)(Brisson, 2013)


\par ==>  Le quantificateur CHAQUE :
\par Chaque lui par contre demande que chaque entité du domaine ait été observé isolement
contrairement à tout (Mari et Retoré, 2016). Elle aurait une sémantique distributive et est malvenu dans des phrases génériques, son emploi est plutôt descriptif.

\par
En français comme en logique, l’utilisation de tout équivaut à celui de chaque mais la réciproque ne fonctionne pas car dans la plupart des énoncés avec chaque, tout induit une relation causale peu évidente ou même d’ailleurs quasi impossible


%A compltr..............................................................................

\vspace{5mm}
QUANTIFICATEURS EXISTENTIELS:
\par La quantification existentielle affirme ou nie l’existence d’au moins une chose.

\par => CERTAIN :
« Certains est traité comme indéfini au sens strict dans de nombreuses approches, alors que
plusieurs auteurs l’excluent de cette classe » (Corblin, 1997). 
« Il semblerait que certains a en général une prédilection marquée pour les portées large et intermédiaire » (ibid.).

\par => UN :

\par Dans la langue française le sens du quantificateur un n’est jamais compris dans les phrases, sa compréhension est toujours associé au chiffre un. Son interprétation réelle équivaut à « au moins » dans les phrases dans lesquelles il est utilisé.
\par 
Par exemple :
Pierre a un diplôme en Informatique doit être compris Pierre a au moins un diplôme en informatique
et non Pierre a un seul diplôme en informatique.


%A compltr..............................................................................




En français comme en logique, l’utilisation de tout équivaut à celui de chaque mais la réciproque
ne fonctionne pas car dans la plupart des énoncés avec chaque, tout induit une relation causale peu
évidente ou même d’ailleurs quasi impossible.

\vspace{5mm}
%......................pas sur....................................................
\par On peut lire une formule/énoncé de différentes                             %.
façons :                                                                        %.
par exemple, concernant le quantificateur universel on peut le lire comme tel : %.
quelque soit, pour tout, pour chaque, etc.                                      %.



\par Quelle est donc la lecture des énoncés la plus répondu dans le domaine mathématique ?
%.................................................................................
\vspace{5mm}
\par Afin de répondre à cette question, il est intéressant de programmer une famille de questionnaire où nous nous appuierons sur les différentes interprétations des quantificateurs en termes de logique dans les phrases, l’ambiguïté dans leurs expressions en fonction des mots, suivie par une étude statistique simple afin de connaître les
préférences des locuteurs.






\vspace{5mm}
\par REFERENCES : \par
MARI, A. et RETORE, C. (2016) Conditions d’assertion de chaque et de tout et règles de
déduction du quantificateur universel. Travaux de linguistique, p. 89 à 106.
\par
BRISSON, C. (2013) Plural, « all », and the nonuniformity of collective predication. In
Linguistics and Philosophy, p. 129 – 184.
\par
KLEIBER, G. (2011) La quantification universelle en trio : tous les, chaque et tout. Studii de
lingvistica, p. 139 – 157.

\par DOWTY, G. (1987) Collective predicate, distributive predicate and all 

\end{document}